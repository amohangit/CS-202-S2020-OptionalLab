% Add the honor code statement here. 
% Comments are added in Latex using the percentage sign. 

\documentclass{article}
\usepackage[utf8]{inputenc}
\usepackage{verbatim}
\usepackage{fancyvrb}
\usepackage{graphicx}
\usepackage{latexsym}
\usepackage{color}
\usepackage{listings}

\usepackage{algorithm2e}
\usepackage{algorithmic}

\title{exp-tracker1}
\author{amohan }
\date{January 2020}

\begin{document}
\begin{itemize}
\item[]
\textbf{Algorithm:} ExpenseTracker(E)
\item[]
\textbf{Input:} An n-element array E of expenses each day.
\item[]
\textbf{Output:} An n-element array A of values such that 
A[i] is the average of elements E[0], E[1], ... , E[i]
\item[]
\scalebox{1}{
\begin{algorithm}[H]
\begin{algorithmic}[1]
\STATE Initialize $a, i, j = 0$
\FOR{$(i=0; i<n; i=i+1)$}
\STATE $a \gets 0$
\FOR{$(j=0; j<i; i=i+1)$}
\STATE $a \gets a + E[j]$
\ENDFOR
\STATE $A[i] \gets a/i+1$
\ENDFOR
\STATE \textbf{return A}
\end{algorithmic}
\label{alg:seq}
\end{algorithm}
}
\end{itemize}
\end{document}
